\chyph
\parskip=0.3cm
\parindent=0pt

{\bf Interaktivní playlist} (Jakub Maroušek, Jakub Hančin)

Cílem našeho projektu je vyrobit aplikaci pro přehrávání hudby s 2D grafickým
playlistem, v němž budou skladby seřazené podle jejich hudebních vlastností,
jako je např. tempo, dynamika, \uv{tancovatelnost}, apod.

Základem aplikace je vylepšený extraktor příznaků ze skladeb, který bude
vytvořený na základě některé volně dostupné knihovny pro extrakci.
Nízkoúrovňové příznaky (energy, magnitude spectrum, spectral flux apod.), které
nám knihovna extrahuje, budeme upravovat, aby nám vznikly vysokoúrovňové
vlastnosti, které už půjdou přímo v programu využít.

Samotná aplikace bude grafická a bude mít schopnost přijmout nové hudební
soubory, přidat je do databáze a spustit extrakci vlastností. Playlist je
čtvercová plocha se dvěma osami, z nichž každá reprezentuje jeden příznak (to
si může zvolit uživatel). Na ploše se zobrazují skladby náhodně vybrané z
databáze a umístěné podle hodnoty příznaků. Skladby se v playlistu v čase
střídají.

Počet implementovaných příznaků nemůžeme dopředu slíbit, závisí na naší
schopnosti je efektivně extrahovat. Použité technologie zatím nejsou vybrané,
chceme si nejdřív zjistit, které příznaky budeme pro extrakci potřebovat, a na
základě toho vybrat knihovnu.

\bye
